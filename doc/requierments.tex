\documentclass[utf8]{report}
\usepackage[utf8]{inputenc}
\usepackage[english,russian]{babel}
\usepackage{times}
 
\begin{document}
 
% Article top matter
\title{UltraBird: требования}
\author{Blokhin Yuri\\
        \texttt{ultrablox@gmail.com}}
\date{\today}
\maketitle

\tableofcontents
 
\begin{abstract}
Данный документ описывает требования к составным элементам системы автопилота на всех уровнях. 
\end{abstract}
 
\chapter{Требования верхнего уровня}

\section{Общие требования}

\begin{enumerate}
  \item СА должна ориентироваться в пространстве на основании показаний бортовых датчиков, а также дополнительных сенсоров, установленных на носителе.
  \item Должна быть поддержка управления габаритными огнями, устанавливаемыми на носителе.
  \item СА должна быть оснащёна оборудованием для двусторонней коммуникации со стационарной управляющей станцией.
  \item Электронные компоненты должны питаться от электрических аккумуляторов. При этом, аккумуляторы должны быть легко сменными.
  \item СА должна уметь подзаряжать свой аккумулятор от мощного аккумулятора носителя.
  \item Оборудование должно быть способно работать в широком диапазоне погодных условий (подразумевается по крайней мере наличие влагозащищенности, усойчивости к ветру).
\end{enumerate}

\subsection{Дополнительные требования для поддержки летательных аппаратов}

\begin{enumerate}
    \item СА должна поддерживать управление приспособлением для стоянки на земле и другой горизонтальной поверхности (посадочные шасси) в случае наличия соответствующей на носителе.
    \item СА должна уметь управлять дополнительными манипуляторами, которое, например, сможет производить операцию над переносимым грузом для задачи переноски груза.
\end{enumerate}

\section{Дополнительные требования в рамках решения конкретных задач}

\subsection{Задача прямого наблюдения}
\begin{enumerate}
  \item БПЛА должен снабжаться панорамными камерами видимого спектра для захвата видео в дневных условиях, или/и камерами ИК-диапазона для ведения наблюдения в ночных условиях.
  \item БПЛА должен быть способен передавать видео с бортовых камер на управляющую станцию.
  \item Предполагается, что БПЛА может вести наблюдение, находясь вне пределов прямой видимости оператора в ясную погоду (на расстоянии более 1км от оператора).
\end{enumerate}

\subsection{Задача фото/видео-разведки}
\begin{enumerate}
  \item БПЛА должен удовлетворять всем требованиям, для выполнения задачи прямого наблюдения.
\end{enumerate}

\subsection{Задача переноса полезного груза}
\begin{enumerate}
  \item Для переноса груза БПЛА должен снабжаться управляемым захватом для груза. В качестве альтернативы допускается использование неуправляемого подвеса (при этом груз закрепляется на БПЛА человеком).
  \item Для осуществления посадки БПЛА должен быть оснащён датчиками расстояния до поверхности земли.
  \item Решить об установке короткодействующих датчиков расстояния.
\end{enumerate}

\subsection{Задача агрессивного маневрирования}
\begin{enumerate}
  \item БПЛА должен снабжаться двигателями с достаточным запасом мощности для выполнения резких манёвров.
\end{enumerate}

\subsection{Задача геодезии}
\begin{enumerate}
  \item БПЛА должен снабжаться управляемым источником света, который может быть либо включен, либо выключен. При этом он должен быть закреплен статически на БПЛА.
\end{enumerate}

\section{Требования к модульности}

\subsection{Общие требования}
\begin{enumerate}
  \item Модульность. СА должна делится на модули так, чтобы можно было заменить один модуль, не изменяя при этом железо / программу на других.
  \item Размеры. Размеры мозгов должны быть минимально возможными. При этом они должны быть достаточно большими для того, чтобы не нарушать простоту сборки / замены модулей.
  \item Все модули должны иметь одинаковый интерфейс (находиться на общей шине). Необходим унифицированный протокол общения между модулями. Провода питания и заземления конструктивно объединены со шлейфом шины связи.
  \item Необходимо наличие выведенного на специальный разъем интерфейса для прошивки имеющихся на модуле контроллеров.
  \item Модули должны иметь одинаковый размер в плоскости платы.
  \item Модули должны иметь конструктивные элементы для установки.
  \item Модули должны принимать информационные запросы. При этом, приняв запрос, модуль должен его обработать, сформировать ответ, и отправить ответ запросившему модулю.
  \item При поступлении запроса на модуль, исполнение которого подразумевает отправку простого отчета о выполнении (выполнен / не выполнен), модуль должен обрабатывать поступивший запрос и отправлять соответствующий отчет. Такие запросы будут называться простыми.
  \item Конкретные размеры. Так как на данный момент в качестве центрального модуля выбрана плата BeagleBone, то данная плата и должна определять размеры остальных модулей.
  \item Индикация состояния. На модуле должны находиться светодиоды, которые показывают в каком состоянии находится модуль. Набор состояний должен включать: нормальное, обнаружены несерьезные ошибки, обнаружены серьезные ошибки. За управление диодами должен отвечать контроллер, без которого функционирование модуля не возможно в принципе. Диоды служат органом визуальной диагностики модулей.
  \item Кроме индикации диодами, состояние должно запрашиваться по протоколу взаимодействия.
\end{enumerate}

\subsection{Детальные требования к модулям}

\subsubsection{Модуль шасси}
\begin{enumerate}
  \item Модуль должен иметь возможность управления различным модельным электро-оборудованием посредством ШИМ сигнала. Вследствие унификации разъемов и типа ШИМ, используемых в модельной техники, расчет должен быть на них.
  \item Модуль должен иметь 16 независимых выходных ШИМ-каналов.
  \item Модуль должен иметь возможность обработки на вход 8 ШИМ-каналов, например для прямого управления с классической радиоаппаратуры.
  \item Модуль должен иметь программный интерфейс, позволяющий:
  \begin{enumerate}
    \item Установить значение ШИМ определенного канала
    \item Установить значение ШИМ всех каналов
    \item Включить или выключить определенный выходной канал
  \end{enumerate}
  \item Так как плата является частью большого "бутерброда", разъемы должны выходить вбок. Таким образом, будет возможно подключать требуемые каналы, если модуль будет находиться внутри стопки.
\end{enumerate}

\subsubsection{Сенсорный модуль}
\begin{enumerate}
  \item Принимаемая информация
  \begin{enumerate}
    \item Модуль должен обрабатывать запрос на данные о текущем местоположении (высота, широта, долгота, ориентация, производные по вышеперечисленным параметрам).
    \item Модуль должен принимать запрос на диагностику датчиков.
    \item Модуль должен принимать простой запрос на сброс летной информации. При этом, модуль содержит информацию о текущем местоположении, но из-за погрешностей датчиков или других внешних факторов иногда может потребоваться привести память аппарата в состояние, когда он не знает где находится - для этого и служит данный запрос.
    \item Модуль должен обрабатывать простой запрос на включение / выключение определенных датчиков.
    \item Модуль должен обрабатывать запрос на данные о пройденной траектории за указанный интервал времени.
  \end{enumerate}
  \item Отправляемая информация
  \begin{enumerate}
    \item Модуль должен по запросу предоставлять приведенную информацию с датчиков (высота, широта, долгота, ориентация, производные по вышеперечисленным параметрам) (ответ на запрос 1).
    \item Модуль должен предоставлять результат диагностики датчиков, производить диагностику и предоставлять отчет (ответ на запрос 2).
    \item Модуль должен предоставлять данные о пройденной траектории за запрошенный интервал (ответ на запрос 5).
  \end{enumerate}
  \item Требования к датчикам ускорения (акселерометры)
  \begin{enumerate}
    \item Ускорение должно фиксироваться по 3-м перпендикулярным осям.
    \item Предельно измеряемые значения должны быть не меньше 1g (свободное падение аппарата должно фиксироваться датчиками).
    \item Должно фиксироваться ускорение при полёте вертикально вниз на полной мощности двигателей. Оценка максимального суммарного ускорения 2.5g (уточнить в обязательном порядке).
  \end{enumerate}
  \item Требования к датчикам вращения (гироскопы)
  \begin{enumerate}
    \item Вращение должно фиксироваться по всем 3-м перпендикулярным осям.
    \item Должно фиксироваться максимальное вращение вокруг вертикальной оси, создаваемое самим аппаратом, за счет момента импульса винтовых двигателей.
    \item Должно фиксироваться максимальное вращение вокруг горизонтальных осей, производимое на максимальной тяге двигателей, направленных на вращения аппарата.
  \end{enumerate}
  \item Требования к датчикам магнитного поля (магнитометрам)
  \begin{enumerate}
    \item Магнитное поле должно фиксироваться по всем 3-м осям инерциальной системы БПЛА.
    \item Спутниковые датчики местоположения (GPS)
    \item При перемещении на расстояние, величина которого сравнима с погрешностью GPS, данные с датчика носят исключительно справочный характер.
    \item Допускается временная потеря связи. При этом, навигация должна происходить по инерциальной системе.
    \item Для инициализации текущего местоположения после включения, в случае необходимости полета на дальние расстояния, может использоваться как GPS,  так и ручное задание местоположения.
    \item Допускается наличие выносной внешней антенны для повышения качества приема.
  \end{enumerate}
  \item Требования к датчикам высоты (альтиметр)
  \begin{enumerate}
    \item Особые требования отсутствуют.
  \end{enumerate}
  \item Требования к контроллеру модуля
  \begin{enumerate}
    \item Контроллер должен собирать информацию с датчиков с определенным интервалом времени.
    \item Контроллер должен осуществлять коммуникацию навигационного модуля с другими.
    \item Контроллер должен обеспечивать сохранение текущей траектории на носитель информации, а так же показания датчиков и время фиксации.
  \end{enumerate}
\end{enumerate}

\subsubsection{Силовой модуль}
\begin{enumerate}
  \item Силовой модуль должен обеспечивать питание всего БПЛА.
  \item Силовой модуль должен быть посредником между аккумуляторами и любыми другими электронными компонентами БПЛА.
  \item Принимаемые запросы
  \begin{enumerate}
    \item Модуль должен принимать запрос на уточнение информации об контроллируемых двигателях.
    \item Модуль должен принимать запрос на изменение мощности работы двигателей.
    \item Модуль должен принимать запрос на включение / отключение питания указанных модулей.
    \item Модуль должен принимать запрос на уточнение заряженности аккумуляторов.
  \end{enumerate}
  \item Отправляемая информация
  \begin{enumerate}
    \item Модуль должен сообщать о положении требуемого двигателя.
    \item Модуль должен сообщать о скорости работы требуемого двигателя.
    \item Модуль должен сообщать о текущих параметров заряженности аккумуляторов.
    \item Модуль должен сообщать о прогнозируемом времени остатка работы аккумуляторов. В случае, если прогноз не возможен, модуль должен вместо времени работы сообщить о невозможности его вычисления. Для этого допускается использование информации как о текущем напряжении на аккумуляторе, так и статистическая информация о реалном времени работы аккумулятора, и другие данные, которые могут быть загружены в модуль о кокнкретном аккумуляторе.
  \end{enumerate}
\end{enumerate}

\subsubsection{Радиомодуль}
\begin{enumerate}
  \item Принимаемая информация
  \begin{enumerate}
    \item Модуль должен принимать запрос от вычислительного модуля на отправку блока информации. При этом информация может классифицироваться как обычная (при этом, если ее отправка невозможна в течении установленного времени, то она забывается) и обязательная (информация будет отправлена, как только появится возможность ее передать; до тех пор она будет занесена во временный накопитель информации).
    \item Модуль должен принимать запрос от видео модуля на отправку кадра видео информации, связанной с пространственной координатой и временем.
    \item Модуль должен принимать запрос на уточнение уровня сигнала (возможности связи с управляющей станцией).
  \end{enumerate}
  \item Отправляемая информация
  \begin{enumerate}
    \item Модуль должен отправлять отчет об отправке блока информации запросившему, либо передавать невозможность (ответы на запросы 1, 2).
    \item Модуль должен отправлять блок информации на вычислительный модуль, в случае его поступления на радио-приемник.
    \item Модуль должен отправлять информацию об уровне сигнала и возможности связи с базой (ответ на запрос 3).
  \end{enumerate}
  \item Передача информации по радио каналу
  \begin{enumerate}
    \item Модуль должен иметь 2 канала радиопередачи, на одном должна достигаться высокая скорость передачи, но низкая дальность, на другом - высокая дальность при низкой скорости передачи. При этом, каналы могут работать одновременно, но не должны мешать друг другу на физическом уровне.
    \item При поступлении блока информации, модуль должен сам принимать решение о том, по какому каналу информации его передавать.
    \item Информация должна кодироваться так, чтобы была невозможна расшифровка в случае перехвата, а так же невозможна ложная подмена управляющей станции станцией злоумышленника.
  \end{enumerate}
\end{enumerate}

\subsubsection{Вычислительный модуль}
\begin{enumerate}
  \item Данный модуль должен исполнять логику полетов.
  \item Требования к носителю информации
  \begin{enumerate}
    \item Носитель информации должен обеспечивать достаточное количество места, которое может понадобиться для траектории рассчитанное на максимальное время полета аппарата.
    \item Носитель информации не должен сбрасывать информацию при плановом отключении питания.
    \item Информация не должна теряться при неожиданном отключении питания.
    \item Память должна быть перезаписываемой.
  \end{enumerate}
\end{enumerate}

\subsubsection{Видео модуль}
\begin{enumerate}
  \item Принимаемая информация. Модуль должен принимать запросы на получение видео-информации.
  \item Отправляемая информация. Модуль должен отправлять блоки закодированной видео-информации с дополнительной информацией.
  \item Обработка видео информации
  \begin{enumerate}
    \item Модуль должен контролировать не менее одной видеокамеры, установленной на борту БПЛА.
    \item Модуль должен по запросу уметь снимать видео-поток с видео камеры, после чего сжимать ее, а затем формировать блок видео-информации с добавлением поясняющей информации (например, с какой камеры какое изображение получено), который отправлять запросившему.
  \end{enumerate}
\end{enumerate}

\section{Требования к управляющей стационарной станции}

\subsection{Аппаратное обеспечение}
\begin{enumerate}
  \item Управление БПЛА должно производиться с помощью компьютера, к которому подключается специальный модуль связи.
  \item Модуль связи должен иметь USB-интерфейс для связи с компьютером.
  \item Модуль связи должен иметь необходимые трансиверы для коммуникации с БПЛА.
\end{enumerate}

\subsection{Программное обеспечение}
\begin{enumerate}
  \item Пользователь должен иметь возможность задания полетного задания. Задания должны быть нескольких типов.
  \item В любой момент пользователь должен иметь возможность прервать исполнение полетного задания и остановить аппарат. При этом, аппарат должен переходить в режим прямого подчинения.
  \item В любой момент пользователь должен иметь возможность отозвать БПЛА. При этом БПЛА должен произвести возврат в точку взлета и приземлиться.
  \item Пользователь должен иметь возможность просмотреть весь переданный видеопоток за время полета. При этом, он должен видеть его в виде набора фотоизображений, закрепленных в пространстве вирутальной карты так, как их видел БПЛА во время захвата. 
\end{enumerate}

\subsection{Мониторинг состояния БПЛА}
При работе управляющего программного обеспечения, должен производиться мониторинг состояния БПЛА. Пользователь должен в любой момент времени ее видеть на экране. К ней относятся:
\begin{enumerate}
  \item Данные о местоположении (3 измерения скорости, 3 измерения координаты, 3 измерения поворота - однозначное положение в пространстве). При этом, необходимо чтобы пользователь видел траекторию движения БПЛА на виртуальной карте, чтобы он мог пространственно представлять движение аппарата
  \item Уровень заряда аккумуляторов
  \item Интенсивность работы двигателей
  \item Исправность оборудования
\end{enumerate}

\subsection{Режим прямого подчинения}
Данный режим служит для решения задачи прямого наблюдения. Пользователь должен управлять БПЛА с компьюьтера, как с пульта ДУ. При этом пользователь может исполнять следующие команды:
\begin{enumerate}
  \item Включить моторы и приготовиться к полету
  \item Изменить высоту (производить подъем / спуск)
  \item Производить движение вперед
  \item Установить предельную скорость движения
  \item Производить поворот вокруг вертикальной оси Z
  \item Лететь в заданную точку
  \item Приземлиться
\end{enumerate}

\subsection{Полетные задания}
\begin{enumerate}
  \item Полет в заданную точку. Данное полетное задание служит для перемещения аппарата в нужную точку. Для пользователя назначение данного задания должно происходить как можно проще, например щелчком мыши по виртуальной карте. При этом не важно, каким маршрутом полетит БПЛА, важно чтобы БПЛА прилетел в точку назначения как можно быстрее.
  \item Полет по маршруту. Данное полетное задание служит для перемещения аппарата в нужную точку, если есть требования к маршруту. Пользователь должен при этом нарисовать маршрут, после чего приказать исполнять его. Аппарат должен спокойно его придерживаясь, прилететь в заданную точку назначения. При покидании зоны связи, аппарат должен прервать выполнение задания и вернуться в зону связи.
  \item Отработка заданной траектории. Данное полетное задание служит для исполнения задачи маневрирования. Пользователь должен изобразить траекторию к исполнению, приказать начать ее исполнение, после чего аппарат должен ее исполнить. При этом, если пользователь прикажет остановить ее исполнение, БПЛА должен зависнуть в ближайшем устойчивом положении, как только у него появится такая возможность (например, в случае если он исполняет траекторию с переворотами в опасной зоне, то не всегда он сможет тут же зависнуть).
  \item Слежение за объектом. Данное полетное задание может быть выдано из режима прямого подчинения и служит для выполнения задачи слежения. Для этого пользователь должен выделить распознанный БПЛА образ, и приказать следовать за ним. Сразу после подтверждения, БПЛА должен приступить к выполнению задания. В случае потери цели, БПЛА должен вернуться в зону связи и доложить о провале.
  \item Разведка. Данное полетное задание служит для выполнения задачи разведки. Оно эквивалентно полету по маршруту, но в отличие от него, оно не будет остановлено в случае выхода из зоны связи. Так же, помимо маршрута полета, задается зона, которую необходимо разведать. БПЛА должен скорректировать маршрут исходя из указанной зоны и сфотографировать ее . После возвращения в зону связи, с наивысшим приоритетом будет передана развединформация, и только потом будет передаваться текущий видеопоток.
\end{enumerate}

\chapter{Функциональные требования}

\section{Общие требования к модулям СА}

\subsection{Статус и индикация}

\begin{enumerate}
  \item Модуль может находитсья в трех состояниях: нормальное, произошли небольшие ошибки, произошли критические ошибки.
  \item Для индикации состояния используется красный светодиод. Если красный диод не горит вообще - это нормальное состояние. Если он мигает редко, это говорит о наличии незначительных ошибок. Если он мигает часто - значит произошли критические ошибки.
  \item Статус модуля может быть запрошен по шине взаимодействия.
\end{enumerate}

\subsection{Управление}

\begin{enumerate}
  \item \textbf{Запрос:} Получить статус модуля. \textbf{Ответ:} Текущий статус, а также номер ошибки, вызвавший ненормальное состояние (если состояние действительно ненормальное).
\end{enumerate}

\section{Модуль шасси}

\begin{enumerate}
  \item Каждый выходной разъем должен состоять из 3-х коннекторов: земля, питающее напряжение (4.6-6.0В), и ШИМ.
  \item Параметры выходного и входного ШИМ-сигналов должны быть следующими: размер "кадра" - 20 - 22.5мс, размер полезного значения 1-2мс. При этом, среднее положение устройства определяется шириной полки в 1.5мс.
  \item Анализ входных ШИМ сигналов должнен осуществляться на основе таких же параметров.
  \item Модуль должен иметь 16 выходных и 8 входных ШИМ-сигналов.
  \item Нумерация выходных портов должна начинаться с 1 и далее инкрементально.
  \item Нумерация входных портов должна начинаться с 0 и далее инкрементально.
  \item Модуль должен работать в 3-х режимах: нормальный, прямой, комбинированный.
  \item Модуль должен иметь светодиоды, на основе которых осуществляется индикация текущего режима работы. Зеленый - нормальный режим, синий - прямой, желтый - комбинированный.
  \item При инициализации модуля должен производиться опрос имеющихся устройств и после этого определяться число доступных входных и выходных каналов.
\end{enumerate}

\subsection{Режимы работы}

\textbf{Нормальный}. Модуль устанавливает заданные вычислительным модулем значение на выходные каналы.

\textbf{Прямой}. Модуль устанавливает значение выходного канала i-го канала (i >= 1) равное значению i-го входного канала. 0-й канал при этом не участвует в процессе. При этом, модуль позволяет читать значения входных каналов, но не позволяет выставлять значения выходных запросами со стороны радиомодуля.

\textbf{Комбинированный}. Модуль регулярно проверяет значение входного 0-го канала, который интерпретируется как бинарный переключатель. Если переключатель выключен, то модуль работает в нормальном режиме. 

\subsection{Управление}

Управление модулем должно осуществляться на основе запроса/ответа со стороны вычислительного модуля. Рассмотрим возможные команды:

\begin{enumerate}
  \item \textbf{Запрос:} Установить значение ШИМ определенного канала. \textbf{Ответ:} результат выполнения (успешно, либо нет и объяснение). Например, может оказаться невозможно, если модуль находится в прямом режиме, и идет попытка установки "прозрачного" канала. При этом, команда меняет только один канал, все другие остаются нетронутыми.
  \item \textbf{Запрос:} Установить значение ШИМ всех каналов \textbf{Ответ:} результат выполнения (успех, частичный успех, провал). Используется, если нужно сразу выставить значения всех доступных каналов.
  \item \textbf{Запрос:} Получить значения входных каналов. \textbf{Ответ:} множество упорядочных вещественных чисел по каналам.
\end{enumerate}

\subsection{Коды ошибок}

\begin{enumerate}
  \item Какой-то компонент модуля стал недоступен.
\end{enumerate}


\section{Сенсорный модуль}

Прежде всего, необходимо определить набор компонентов, обязательный к установке на модуль.

\begin{enumerate}
  \item Необходимо наличие следующих базовых датчиков: акселерометр, гироскоп, магнетометр (либо комбинированный Motion Processing Unit)
  \item Требуется наличие барометрического альтиметра для определения высоты
  \item Требуется наличие GPS-датчика с выносной антенной
\end{enumerate}

\subsection{Управление}

Управление модулем должно осуществляться на основе запроса/ответа со стороны вычислительного модуля. Перечислим возможные команды:

\begin{enumerate}
  \item \textbf{Запрос:} Получить текущие показания указанных датчиков. \textbf{Ответ:} множество показаний датчиков. Данная команда используется системой стабилизации, а также для мониторинга показаний датчиков.
  \item \textbf{Запрос:} Получить текущие координаты. \textbf{Ответ:} точка в пространстве. Данная команда используется при сохранении траектории БПЛА.
\end{enumerate}

\section{Радио-модуль}

В первом приближении, радио-модуль служит для обмена служебными сообщениями и не способен работать с видео.

\begin{enumerate}
  \item Модуль должен иметь очередь входящих и исходящих сообщений, что даст возможность обрабатывать ситуации с потерей связи с управляеющей станцией.
  \item Модуль должен иметь выносную антенну для улучшения дальности и качества связи
  \item Модуль должен разделять сообщения на важность: доставить в любом случае, доставить в случае наличия связи
  \item Модуль должен обеспечивать гарантию доставки для важных сообщений (например, передача полетного задания)
\end{enumerate}

\subsection{Структура сообщения}

Управляющая станция и радиомодуль взаимодействуют по принципу запрос-ответ со стороны управляющей станции. Существуют следующие типы сообщений:

\begin{enumerate}
  \item Важные. Например, описание полетного задания. Такие сообщения должны быть доставлены в любом случае, даже если на данный момент связи с СА нету. В таком случае сообщение должно встать в очередь, и быть доставлено, как только появится такая возможность. При этом, должен быть обеспечен контроль, что сообщение действительно принято и не искажено.
  \item Средней важности. Например, значение каналов в режиме прямого управления через управляющую станцию. Если какие-то сообщения потеряются - ничего страшного. В любой момент времени важно только последнее сообщение.
  \item Низкой важности. Например, регулярный опрос датчиков с целью мониторинга состояния СА. Если связи нет - то сообщения вообще можно не посылать. Если канал занят чем-то более важным - то их тоже можно пропустить.
\end{enumerate}

\subsection{Управление}
\begin{enumerate}
  \item \textbf{Запрос:} Получить текущие входящие сообщения. \textbf{Ответ:} множество входящих сообщений. У сообщений есть порядковый номер, по которому управляющая станция ожидает ответ.
  \item \textbf{Запрос:} Отправить сообщение на управляющую станцию. \textbf{Ответ:} Тело сообщения и идентификатор в ответ на которое сформировано данное сообщение.
\end{enumerate}

\section{Управляющая станция}

\begin{enumerate}
  \item Питание на станцию должно подаваться посредством USB-порта или через отдельный разъем.
  \item Станция должна предоставлять ethernet-интерфейс для взаимодействия.
  \item Станция должна держать 2 программных сервера - через один осущствляется взаимодействие путем обмена управляющими командами, а через другой отдача потока видео.
\end{enumerate}

\section{Программное обеспечение оператора}

Управление БПЛА производится с компьютера с помощью специального ПО (программное обеспечение оператора - далее ПОО). ПО связывается с управляющей станцией, и все действия производит через нее.

\begin{enumerate}
  \item ПОО должно представлять из себя оконное приложение.
  \item ПОО должно иметь некий индикатор наличия соединения с управляющей станцией.
  \item Функциональность ПОО должна быть разбита по различным опционально отображаемым панелям (виджетам), при этом должно быть основное центральное окно (всегда видимое).
  \item Все управление БПЛА должно осуществляться на основе обращения к управляющей станции с помощью соответствующего API через унифицированный интерфейс.
  \item Среда должна поддерживать легко-подключаемые плагины, для возможности расширения функциональности среды для определенных задач/носителей.
  \item Второстепенные виджеты должны быть реализованы по принципу плагинов для однородности компонентов среды.
\end{enumerate}

Требования к основному окну.

\begin{enumerate}
  \item Должна отображаться карта с возможностью масштабирования (zoom, pan). Возможно, в 3д (тогда к операцаиям view добавляется поворот)
  \item Должна отображаться пройденная траектория аппаратом в виде некоторой линии.
  \item Должна отображаться карта препетствий, построенная аппаратом.
  \item Должны отображаться показания определенных основных датчиков (либо сводные показатели).
  \item Должно отборажаться состояние модулей БПЛА, либо информация по состоянию неисправных/проблемных модулей.
\end{enumerate}

Требования к плагинам.

\begin{enumerate}
  \item Плагин должен иметь собственный графический интерфейс в виде виджета.
  \item Каждому плагину передается API-объект доступа к управляющей станции, через который он может обмениваться с ней командами.
  \item Можно попробовать сделать возможным поддержку плагинов на интерпретируемых языках (типа руби, питона), если возникнет необходимость.
\end{enumerate}

\begin{thebibliography}{9}
%The \bibitem is to start a new reference.  Ensure that the cite_key is
%unique.  You don't need to put each element on a new line, but I did
%simply for readability.
    \bibitem{lamport94}
      Leslie Lamport,
      \emph{\LaTeX: A Document Preparation System}.
      Addison Wesley, Massachusetts,
      2nd Edition,
      1994.
 
\end{thebibliography}
 
\end{document}
